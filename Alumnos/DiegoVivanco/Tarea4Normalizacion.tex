\documentclass[spanish]{article}
\usepackage[T1]{fontenc}
\usepackage[utf8]{inputenc}
%\usepackage[spanish]{babel}
\usepackage[spanish,activeacute,mexico]{babel}
\usepackage[margin=2.5cm]{geometry} %tamaño de la hoja
\usepackage{graphicx} % Paquete para imágenes
\usepackage{amsmath} % Paquete para usar más funciones de matemáticas
\usepackage{amssymb} %Para el therefore
\usepackage{enumitem}
 \usepackage[x11names, table]{xcolor}
\begin{document}
%Datos del equipo
\title{
\centering{\includegraphics[width=1\linewidth]{escudos.png}\\[1cm]}\\
UNIVERSIDAD NACIONAL AUTONOMA DE MEXICO\\
\vfill
Facultad de Ingenieria\\
\vfill
{\bfseries TAREA 4}
\vfill
{\bfseries Normalización de Bases de Datos }
\vfill
\\Grupo 5\
\vfill
Semestre 2020-2\\
\vfill
BASES DE DATOS
\vfill
Profesor: Ing. Fernando Arreola Franco}
\vfill
% Agreguen sus nombres :v
\author{\textbf{Integrantes}:\\
Vivanco Quintanar, Diego Armando\\}
\date{26 de marzo de 2020}
\maketitle
\newpage
\section{EJERCICIO}

    A partir de la siguiente tabla normalizar hasta la segunda forma.
    
    \begin{table}[ht]
    \centering
	\begin{tabular}{|c|c|c|c|c|c|l|}
	\hline
	\rowcolor{LightBlue2}
	\textbf{staffNo} & \textbf{branchNo} & \textbf{branchAddress} &  \textbf{name} & \textbf{position} & \textbf{hoursPerWeek}\\ \hline
	 S4555 & B002 & City Center Plaza, Seattle, WA 98122 & Ellen Layman & Assistant & 16\\ \hline
	 S4555 & B004 & 16-14th Avenue, Seattle, WA 98128 & Ellen Layman & Assistant & 9\\ \hline
	 S4612 & B002 & City Center Plaza, Seattle, WA 98122 & Dave Sinclair & Assistant & 14\\ \hline
	 S4612 & B004 & 16-14th Avenue, Seattle, WA 98128 & Dave Sinclair & Assistant & 10\\ \hline
	\end{tabular}
	\caption{Tabla con los datos del personal de una empresa.} \label{semi}
	\end{table}
	
	
	Como podemos observar en la tabla \ref{semi} grupos repetidos tanto para staffNo, branchNo, name y position, la forma de evitar estos grupos repetidos es normalizando dicha tabla a primera forma normal.
	Sin embargo en la clase hemos podido ver que al obtener las dependencias funcionales de la tabla se nos facilita llegar a la segunda formal y por consiguiente a la primer forma normal.
	Partiendo de lo anterior vamos a definir lo siguiente:
	\begin{itemize}
	    \item A $=$ staffNo
	    \item B $=$ branchNo
	    \item C $=$ branchAddress
	    \item D $=$ name
	    \item E $=$ position
	    \item F $=$ hoursPerWeek
	\end{itemize}
	Entonces las dependencias las podemos ver de la siguiente manera:\\
	\begin{itemize}
	    \item [$*$] A $->$ \{D, E\}
	    \item [$*$] A,B $->$ \{F\}
	    \item [$*$] B $->$ \{ C\}
	\end{itemize}
	
	Como sabemos la segunda forma normal evita las dependencias parciales, a través de A podemos obtener la infomracion contenida en D y E, a su vez a partir de B podemos visualizar la información de C. Ahora observemos lo siguiente, a partir de la clave primaria compuesta: A y B podemos llegar a F pero tambien partiendo de A o de B podemos llegar a los demas atributos de nuestra tabla por lo que no hay dependencias parciales.
	
	Asi pues las tablas normalizadas en 2FN quedan de la siguiente manera:\\
	
	\begin{table}[ht]
	    \centering
	    \begin{tabular}{|c|c|c|l|}
	    \hline
	    \rowcolor{LightBlue2}
	    \textbf{StaffNo} & \textbf{name} & \textbf{position}\\ \hline
	        S4555 &  Ellen Layman & Assistant\\ \hline
	        S4612 &  Dave Sinclair & Assistan\\ \hline
	    \end{tabular} 
	    \caption{Tabla con los atributos StaffNo, name y position del personal}
	    \label{tab:Personal}
	\end{table}
	
    \newpage
		\begin{table}[ht]
	    \centering
	    \begin{tabular}{|c|c|c|l|}
	    \hline
	    \rowcolor{LightBlue2}
	    \textbf{StaffNo} & \textbf{branchNo} & \textbf{hoursPerWeek}\\ \hline
	        S4555 &  B002 & 16\\ \hline
	        S4555 &  B004 & 9\\ \hline
	        S4612 &  B002 & 14\\ \hline
	        S4612 &  B004 & 10\\ \hline
	    \end{tabular} 
	    \caption{Tabla con la clave primaria compuesta (StaffNo y branchNo) y las horas que labora el personal.}
	    \label{tab:Horas}
	\end{table}
	
	
			\begin{table}[ht]
	    \centering
	    \begin{tabular}{|c|c|c|l|}
	    \hline
	    \rowcolor{LightBlue2}
	    \textbf{branchNo} & \textbf{branchAddress}\\ \hline
	       B002 & City Center Plaza, Seattle, WA 98122\\ \hline
	       B004 & 16-14th Avenue, Seattle, WA 98128\\ \hline
	    \end{tabular} 
	    \caption{Tabla con los atributos branchNo y las horas que labora el personal.}
	    \label{tab:Horas}
	\end{table}
    
\section{Conclusiones}
En este ejercicio se pudo observar que el identificar las dependencias funcionales de la tabla nos permite llegar a la 2FN sin la necesidad de primero pasar por la 1FN y de esa manera hacer una buena normalización de nuestros datos.

\end{document}

