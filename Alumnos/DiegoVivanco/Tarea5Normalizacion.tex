\documentclass[spanish]{article}
\usepackage[T1]{fontenc}
\usepackage[utf8]{inputenc}
%\usepackage[spanish]{babel}
\usepackage[spanish,activeacute,mexico]{babel}
\usepackage[margin=2.5cm]{geometry} %tamaño de la hoja
\usepackage{graphicx} % Paquete para imágenes
\usepackage{amsmath} % Paquete para usar más funciones de matemáticas
\usepackage{amssymb} %Para el therefore
\usepackage{enumitem}
 \usepackage[x11names, table]{xcolor}
\begin{document}
%Datos del equipo
\title{
\centering{\includegraphics[width=1\linewidth]{escudos.png}\\[1cm]}\\
UNIVERSIDAD NACIONAL AUTONOMA DE MEXICO\\
\vfill
Facultad de Ingenieria\\
\vfill
{\bfseries TAREA 5}
\vfill
{\bfseries Normalización de Bases de Datos }
\vfill
\\Grupo 5\
\vfill
Semestre 2020-2\\
\vfill
BASES DE DATOS
\vfill
Profesor: Ing. Fernando Arreola Franco}
\vfill
% Agreguen sus nombres :v
\author{\textbf{Integrantes}:\\
Vivanco Quintanar, Diego Armando\\}
\date{23 de marzo de 2020}
\maketitle
\newpage
\section{EJERCICIO 1}

    A partir de la siguiente tabla normalizar a la primera forma.
    
     \begin{table}[ht]
    \centering
	\begin{tabular}{|c|c|c|c|c|c|l|}
	\hline
	\rowcolor{LightBlue2}
	\textbf{DNI} & \textbf{Nombre} & \textbf{Código$_$Tienda} &  \textbf{Dirección$_$Tienda} & \textbf{Fecha} & \textbf{Turno}\\ \hline
	 33445566 & Paola Martin & 100A & Transmisiones Miliares 70 & M & 02/01/2020\\ \hline
	 44552345 & Laura Sanz & 100A & Transmisiones Miliares 70 & M & 02/01/2020\\ \hline
	 86923456 & Daniel Diaz & 100A & Transmisiones Miliares 70 & T & 02/01/2020\\ \hline
	 33445566 &  Paola Martin & 200B & Periferico Norte 80 & T & 03/01/2020\\ \hline
	 12234456 & Emiliano Lopez & 300C & Av. Universidad 3000 & M & 03/01/2020\\ \hline
	 45678367 & Francisco Monte & 200B & Periferico Norte 80 & M & 03/01/2020 \\ \hline
	 12234456 & Emiliano Lopez & 300C & Av. Universidad 3000 & M & 04/01/2020\\ \hline
	 45678367 & Francisco Monte & 100A & Transmisiones Miliares 70  & M & 04/01/2020 \\ \hline
	 44552345 & Laura Sanz & 100A & Transmisiones Miliares 70 & T & 04/01/2020\\ \hline
	 33445566 & Paola Martin & 200B & Periferico Norte 80 & M & 05/01/2020\\ \hline
	\end{tabular}
	\caption{Tabla original, con el DNI del empleado, nombre del empleado, codigo y direccion de la tienda en el que trabaja el empleado asi como la fecha y el turno del empleado..} 
	\label{original}
	\end{table}
	

    Partiendo de la tarea 3 en donde Normalizamos la tabla \ref{original} a 1FN, obtuvimos las tablas \ref{tabla2} y \ref{tabla3}.
    
    
    
    \begin{table}[ht]
    \centering
	\begin{tabular}{|c|c|c|l|}
	\hline
	\rowcolor{LightBlue2}
	\textbf{DNI} & \textbf{Nombre} & \textbf{Apellido}\\ \hline
	 33445566 & Paola & Martin\\ \hline
	 44552345 & Laura & Saenz\\ \hline
	 86923456 & Daniel & Diaz\\ \hline
	 33445566 & Paola & Martin\\ \hline
	 12234456 & Emiliano & Lopez \\ \hline
	 45678367 & Francisco & Monte \\ \hline
	\end{tabular}
	\caption{Tabla con el DNI del empleado, el nombre y apellido del empleado como atributos.} 
	\label{tabla2}
	\end{table}
	
	
	
	 \begin{table}[ht]
    \centering
	\begin{tabular}{|c|c|c|c|c|l|}
	\hline
	\rowcolor{LightBlue2}
	\textbf{DNI} & \textbf{Codigo$_$Tienda} & \textbf{Dirección$_$Tienda} & \textbf{Fecha} & \textbf{Turno} \\ \hline
	 33445566 & 100A & Transmisiones Miliares 70 & M & 02/01/2020\\ \hline
	 44552345 & 100A & Transmisiones Miliares 70 & M & 02/01/2020\\ \hline
	 86923456 & 100A & Transmisiones Miliares 70 & T & 02/01/2020\\ \hline
	 33445566 & 200B & Periferico Norte 80 & T & 03/01/2020\\ \hline
	 12234456 & 300C & Av. Universidad 3000 & M & 03/01/2020 \\\hline
	 45678367 & 200B & Periferico Norte 80 & M & 03/01/2020 \\ \hline
	 12234456 & 300C & Av. Universidad 3000 & M & 04/01/2020 \\\hline
	 45678367 & 100A & Transmisiones Miliares 70 & M & 04/01/2020 \\ \hline
	 44552345 & 100A & Transmisiones Miliares 70 & T & 04/01/2020\\ \hline
	 33445566 & 100A & Periferico Norte 80 & M & 05/01/2020\\ \hline
	\end{tabular}
	\caption{Tabla con el DNI del empleado y el código de la Tienda como clave principal, los atributos son  la dirección de la tienda, la fecha y el turno del empleado.}
	\label{tabla3}
	\end{table}
	
	
Ahora vamos a definir lo siguiente:\\
\begin{itemize}
    \item A $=$ DNI
    \item B $=$ Nombre
    \item C $=$ Código\_Tienda
    \newpage
    \item D $=$ Dirección$_$Tienda
    \item E $=$ Fecha
    \item F $=$ Turno
\end{itemize}
	
	En la tabla \ref{tabla3} podemos observar que tenemos una dependencia parcial, ya que solo con el código de la tienda podemos acceder a la dirección de la tienda. Para evitar esta dependencia parcial vamos a crear otra tabla con el código de la tienda y la dirección de la tienda, con esto obtenemos la 2FN y las dependencias quedan de la siguiente manera: \\
	\begin{itemize}
	    \item A $->$ \{B\}
	    \item A,C $->$ \{E,F\}
	    \item C $->$ \{D\}
	\end{itemize}
	Las dependencias antes mencionadas se ilustran en las siguientes tablas:\\
	
	\begin{table}[ht]
    \centering
	\begin{tabular}{|c|c|c|l|}
	\hline
	\rowcolor{green}
	\textbf{DNI} & \textbf{Nombre} & \textbf{Apellido}\\ \hline
	 33445566 & Paola & Martin\\ \hline
	 44552345 & Laura & Saenz\\ \hline
	 86923456 & Daniel & Diaz\\ \hline
	 33445566 & Paola & Martin\\ \hline
	 12234456 & Emiliano & Lopez \\ \hline
	 45678367 & Francisco & Monte \\ \hline
	\end{tabular}
	\caption{Tabla con el DNI del empleado, el nombre y apellido del empleado como atributos.} 
	\label{tabla12FN}
	\end{table}
	
    \begin{table}[ht]
    \centering
	\begin{tabular}{|c|c|c|c|c|l|}
	\hline
	\rowcolor{green}
	\textbf{DNI} & \textbf{Código$_$Tienda} & \textbf{Fecha} & \textbf{Turno} \\ \hline
	 33445566 & 100A & M & 02/01/2020\\ \hline
	 44552345 & 100A & M & 02/01/2020\\ \hline
	 86923456 & 100A & T & 02/01/2020\\ \hline
	 33445566 & 200B & T & 03/01/2020\\ \hline
	 12234456 & 300C & M & 03/01/2020 \\\hline
	 45678367 & 200B & M & 03/01/2020 \\ \hline
	 12234456 & 300C & M & 04/01/2020 \\\hline
	 45678367 & 100A & M & 04/01/2020 \\ \hline
	 44552345 & 100A & T & 04/01/2020\\ \hline
	 33445566 & 100A & M & 05/01/2020\\ \hline
	\end{tabular}
	\caption{Tabla con el DNI del empleado y el código de la Tienda como clave principal, los atributos son la fecha y el turno del empleado.}
	\label{tabla2FN}
	\end{table}
    
    
     \begin{table}[ht]
    \centering
	\begin{tabular}{|c|c|l|}
	\hline
	\rowcolor{green}
	\textbf{Código$_$Tienda} & \textbf{Dirección$_$Tienda}\\ \hline
	 100A & Transmisiones Miliares 70 \\ \hline
	 200B & Periferico Norte 80  \\ \hline
	 300C & Av. Universidad 3000  \\\hline
	\end{tabular}
	\caption{Tabla con el DNI del empleado y el código de la Tienda como clave principal, los atributos son la fecha y el turno del empleado.}
	\label{tabla32FN}
	\end{table}
       
    \newpage
   La tercera forma normal nos dice que tenemos que eliminar cualquier columna no llave que sea dependiente de otra columna no llave. Los pasos a seguir son:
   \begin{itemize}
       \item Determinar las columnas que son dependientes de otra columna no llave.
       \item Eliminar esas columnas de la tabla base.
       \item Crear una segunda tabla con esas columnas y con la columna no llave de la cual son dependientes.
   \end{itemize}
  
    En este ejercicio al normalizar a 2FN tambien eliminamos las dependencias transitivas o como se explico anteriormente no hay columnas no llave que dependa de otra columna no llave. Si observamos, en la tabla \ref{tabla12FN} para obtener el nombre de un empleado basta con ver el DNI, en la tabla \ref{tabla2FN} para llegar a las horas que labora tenemos que ver la llave primaria compuesta por el DNI y el código de la Tienda, inalmente en la tabla \ref{tabla32FN} con el código de la Tienda podemos ver la dirección de la tienda. Es decir cada una de las 3 tablas contiene columnas que se pueden acceder unicamente con la clave llame de su respectiva tabla.

\section{EJERCICIO 2}
 A partir de la siguiente tabla normalizar hasta la 3FN.
    
    \begin{table}[ht]
    \centering
	\begin{tabular}{|c|c|c|c|c|c|l|}
	\hline
	\rowcolor{LightBlue2}
	\textbf{staffNo} & \textbf{branchNo} & \textbf{branchAddress} &  \textbf{name} & \textbf{position} & \textbf{hoursPerWeek}\\ \hline
	 S4555 & B002 & City Center Plaza, Seattle, WA 98122 & Ellen Layman & Assistant & 16\\ \hline
	 S4555 & B004 & 16-14th Avenue, Seattle, WA 98128 & Ellen Layman & Assistant & 9\\ \hline
	 S4612 & B002 & City Center Plaza, Seattle, WA 98122 & Dave Sinclair & Assistant & 14\\ \hline
	 S4612 & B004 & 16-14th Avenue, Seattle, WA 98128 & Dave Sinclair & Assistant & 10\\ \hline
	\end{tabular}
	\caption{Tabla con los datos del personal de una empresa.} \label{semi}
	\end{table}
	
	En la tarea 4 normalizamos el ejercicio a 2FN en donde las tablas quedaron de la siguiente manera:\\
	
	\begin{table}[ht]
	    \centering
	    \begin{tabular}{|c|c|c|l|}
	    \hline
	    \rowcolor{green}
	    \textbf{StaffNo} & \textbf{name} & \textbf{position}\\ \hline
	        S4555 &  Ellen Layman & Assistant\\ \hline
	        S4612 &  Dave Sinclair & Assistan\\ \hline
	    \end{tabular} 
	    \caption{Tabla con los atributos StaffNo, name y position del personal}
	    \label{tab:Personal}
	\end{table}
	
		\begin{table}[ht]
	    \centering
	    \begin{tabular}{|c|c|c|l|}
	    \hline
	    \rowcolor{green}
	    \textbf{StaffNo} & \textbf{branchNo} & \textbf{hoursPerWeek}\\ \hline
	        S4555 &  B002 & 16\\ \hline
	        S4555 &  B004 & 9\\ \hline
	        S4612 &  B002 & 14\\ \hline
	        S4612 &  B004 & 10\\ \hline
	    \end{tabular} 
	    \caption{Tabla con la clave primaria compuesta (StaffNo y branchNo) y las horas que labora el personal.}
	    \label{tab:Horas}
	\end{table}
	
	\newpage
			\begin{table}[ht]
	    \centering
	    \begin{tabular}{|c|c|c|l|}
	    \hline
	    \rowcolor{green}
	    \textbf{branchNo} & \textbf{branchAddress}\\ \hline
	       B002 & City Center Plaza, Seattle, WA 98122\\ \hline
	       B004 & 16-14th Avenue, Seattle, WA 98128\\ \hline
	    \end{tabular} 
	    \caption{Tabla con los atributos branchNo y las horas que labora el personal.}
	    \label{tab:Horas}
	\end{table}
	
	Este segundo ejercicio es parecido al ejercicio anterior ya que al tenerlo normalizado en 2FN tanto las dependencias parciales como las transitivas se han eliminado, dando lugar a que en cada una de las tablas las columnas con atributos no llave no pedendan o deriven de otras columnas no llave, es decir para cada tabla las columnas solo dependen de columnas llave.
	
	\section{Conclusiones}\\
	En los ejercicios mostrados observamos que al obtener las dependencias funcionales y obtener la 2FN indirectamente estamos eliminando las dependencias transitivas, lo que significa que obtenemos la tercera forma normal ahorrandonos trabajo de más.
	
\end{document}

