
\documentclass[12pt]{article}
 
\usepackage[margin=1cm]{geometry} 

\begin{document}
 
% --------------------------------------------------------------
%                         Comienzo
% --------------------------------------------------------------
 
\title{Tarea de Bases de Datos\\Normalización}
\author{Esparza Rivera Saúl Abraham\\ Grupo:5}
\date {19 de marzo de 2020}
\maketitle

\textbf{¿Qué es la normalización de bases de datos?\\}
Es el proceso de organizar los datos de una base de datos, valga la redundancia. Debemos tener en cuenta la creación de tablas y las reglas que se usan para definir las relaciones, estas reglas son diseñadas para proteger los datos, y para que la base de datos sea flexible con el fin de eliminar redundancias y dependencias incoherentes.\\ 
 
\textbf{Las bases de datos relacionales se normalizan para:}
\begin{itemize}
\item Evitar la redundancia de los datos.
\item Disminuir problemas de actualización de los datos en las tablas.
\item Proteger la integridad de los datos.
\item Facilitar el acceso e interpretación de los datos.
\item Reducir el tiempo y complejidad de revisión de las bases de datos.
\item Optimizar el espacio de almacenamiento.
\item Prevenir borrados indeseados de datos.
\end{itemize}

\textbf{Requisitos de la normalización\\}
\ Para que las tablas de nuestra BD estén normalizadas deben cumplir las siguientes reglas:
\begin{itemize}
\item Cada tabla debe tener su nombre único.
\item No puede haber dos filas iguales.
\item No se permiten los duplicados.
\item Todos los datos en una columna deben ser del mismo tipo.
\end{itemize}

\end{document}