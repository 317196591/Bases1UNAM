\documentclass[letterpaper,12pt]{article}
\usepackage{chicago}
\usepackage[utf8]{inputenc}
\usepackage[top = 1 cm, bottom = 2 cm, left = 2.5 cm, right = 2.5 cm]{geometry}
\usepackage[spanish]{babel}
\usepackage{amssymb,amsmath}
\usepackage{array,tabularx}
\title{\textbf{Niveles de aislamiento}}
\author{Por: Hernández Francisco Omar }
\usepackage[spanish]{babel}
\date{31 de marzo 2020}
\begin{document}
\maketitle
\thispagestyle{empty}
 El  aislamiento es una propiedad que define cómo y cuándo los cambios producidos por una operación se hacen visibles para las demás operaciones concurrentes en una base de datos.
Si necesita ejecutar consultas específicas con distintos niveles de aislamiento, debe definir distintas conexiones de base de datos.\\\\

Paralograr tener un mayor nivel de aislamiento, un sistema de gestion de base de datos relacional(SGBDR) comunmente hace un bloqueo de los datos e implementa un Control de concurrencia mediante versiones múltiples (MVCC), pero esto puede generar una pérdida de concurrencia.Para ello se necesita añadir lógica adicional al programa que accede a los datos para su funcionamiento correcto.

A continuacion mostramos los niveles de aislamiento desde el nivel más bajo hasta el más alto\\\\
   \begin{enumerate}
       \item Lectura no confirmada\\
       Los cambios efectuados por otras transacciones están disponibles inmediatamente para una transacción.
       \item Lectura confirmada\\
        Una transacción puede acceder sólo a filas confirmadas por otras transacciones.
        \item Estabilidad del cursor\\
        Otras transacciones no pueden actualizar la fila en la que se posiciona una transacción.
       \item Lectura reproducible\\
       Las filas seleccionadas o actualizadas por una transacción no se pueden cambiar por otra transacción hasta que ésta se complete.
       \item Protección fantasma\\
       Una transacción no puede acceder a las filas insertadas o suprimidas desde el inicio de la transacción.
       \item Serializable\\
       Un conjunto de transacciones ejecutado simultáneamente produce el mismo resultado que si se hubiese efectuado de manera secuencial.\\
       Si el aistema gestor de base de datos relacionales hace una implementación basada en bloqueos, la serialización requiere que los bloques de lectura y escritura se liberen al final de la transacción. Del mismo modo deben realizarse bloqueos de rango -sobre los datos seleccionados con SELECT usando WHERE- para evitar el efecto de las lecturas fantasma (ver más abajo).
   \end{enumerate}
   
\end{document}