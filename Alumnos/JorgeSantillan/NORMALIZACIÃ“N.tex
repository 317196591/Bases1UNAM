% Generated by GrindEQ Word-to-LaTeX 
\documentclass{article} %%% use \documentstyle for old LaTeX compilers

\usepackage[english]{babel} %%% 'french', 'german', 'spanish', 'danish', etc.
\usepackage{amssymb}
\usepackage{amsmath}
\usepackage{txfonts}
\usepackage{mathdots}
\usepackage[classicReIm]{kpfonts}
\usepackage[dvips]{graphicx} %%% use 'pdftex' instead of 'dvips' for PDF output

% You can include more LaTeX packages here 


\begin{document}

%\selectlanguage{english} %%% remove comment delimiter ('%') and select language if required


\noindent 

\noindent 

\noindent 

\noindent 

\noindent Santill\'{a}n Gallegos Jorge Refugio  17/03/20

\noindent Tabura S\'{a}nchez Tomihuatzin

\noindent \textbf{NORMALIZACI\'{O}N}.

\noindent Consiste en obtener esquemas relacionales que cumplan unas determinadas condiciones y se centra en las formas normales. Se dice que un esquema esta en determinada forma normal si satisface un conjunto determinado de restricciones.

\noindent 

\noindent �PARA QUE SE UTILIZA LA NORMALIZACI\'{O}N?

\noindent 

\noindent Para el dise\~{n}o de tablas en las que las redundancias de datos se reducen al m\'{i}nimo. Las (1FN, 2FN y 3FN) son las m\'{a}s utilizadas. Entre mayor sea el grado de la forma normal, menor ser\'{a} la redundancia de los datos.

\noindent 

\noindent \textbf{1FN}

\noindent 

\noindent Una relaci\'{o}n est\'{a} en 1� FN si y s\'{o}lo s\'{i} todos los dominios simples subyacentes contienen s\'{o}lo valores at\'{o}micos, \'{o}sea, no tengan atributos multivaluados. 

\noindent 

\noindent Nota: Por la propia definici\'{o}n del modelo de datos relacional, NO se admiten atributos multivaluados. En consecuencia, TODAS las relaciones que aparecen en el modelo de datos relacional est\'{a}n en 1FN.

\noindent \includegraphics*[width=2.34in, height=1.18in, keepaspectratio=false]{image1}\includegraphics*[width=2.34in, height=0.77in, keepaspectratio=false]{image2}R0      R1

\noindent 

\noindent 

\noindent \textbf{}

\noindent \textbf{}

\noindent \textbf{}

\noindent \textbf{}

\noindent \textbf{2FN}

\noindent 

\noindent Una relaci\'{o}n est\'{a} en 2� forma normal s\'{i} y s\'{o}lo s\'{i} est\'{a} en 1� FN y todos los atributos no clave dependen por completo de la llave primaria.

\noindent \includegraphics*[width=1.47in, height=0.39in, keepaspectratio=false]{image3}\includegraphics*[width=2.40in, height=1.40in, keepaspectratio=false]{image4}R0           

  R1

\noindent 

\noindent \includegraphics*[width=2.29in, height=0.84in, keepaspectratio=false]{image5}

  R2

\noindent 

\noindent 

\noindent 

\noindent 

\noindent \eject \textbf{3FN}

\noindent 

\noindent Una relaci\'{o}n est\'{a} en 3� FN s\'{i} y s\'{o}lo s\'{i} est\'{a} en 2� FN y todos los atributos no clave dependen en forma no transitiva de la llave primaria.

\noindent \includegraphics*[width=2.15in, height=1.16in, keepaspectratio=false]{image6}

\noindent \includegraphics*[width=1.46in, height=0.95in, keepaspectratio=false]{image7}R0  R1

\noindent 

\noindent 

\noindent \includegraphics*[width=1.68in, height=0.75in, keepaspectratio=false]{image8}

       R2 

\noindent 

\noindent 

\noindent 

\noindent \textbf{Bibliograf\'{i}a.}

\noindent http://mmedia1.fi-b.unam.mx/material/3454\_aeml670608\_temaiv.bdnormalizacion2020.pdf


\end{document}

