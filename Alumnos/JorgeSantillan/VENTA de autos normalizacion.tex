\documentclass{article} 
\usepackage[english]{babel} 
\usepackage{amssymb}
\usepackage{amsmath}
\usepackage{txfonts}
\usepackage{mathdots}
\usepackage[classicReIm]{kpfonts}
\usepackage[dvips]{graphicx} 


\begin{document}

\noindent VENTA$\mathrm{\{}$num\_Auto, fecha\_Venta, vendedor, comision, descuento$\mathrm{\}}$

\noindent num\_Auto fecha\_Venta

\noindent num\_Auto descuento

\noindent fecha\_Venta descuento

\noindent vendedor  comisi\'{o}n

\noindent un auto puede ser vendido por muchos vendedores

\noindent  [num\_Auto,vendedor](PK)

\noindent Num\_Auto, fecha\_Venta, descuento, vendedor, comisi\'{o}n (tabla 1FN)

\noindent [num\_Auto, vendedor], [vendedor,comision], [num\_Auto,fecha\_Venta,descuento] (tablas 2FN)

\noindent [num\_Auto, vendedor, fecha\_Venta,], [vendedor,comision], [num\_auto,fecha\_Venta,descuento] (tablas 3FN)

\noindent 

\noindent est\'{a} en primera forma normal 

\noindent la segunda forma normal consiste en no tener columnas dependientes de una llave primaria , as\'{i} que separ\'{e} vendedor y num\_Auto y cada una de esas tablas con las columnas con las que se puede conocer con el id de cada una.

\noindent 

\noindent La tercera forma normal tengo dudas, ya que en mis tablas desde la segunda forma ya no encontr\'{e} transitividad hacia otros y para cumplir con lo de un auto puede ser vendido por muchos vendedores, a la hora de realizar la venta y que no se haga repetido un campo agregu\'{e} la fecha , lo que har\'{i}a \'{u}nica a esa llave de venta.

\end{document}
