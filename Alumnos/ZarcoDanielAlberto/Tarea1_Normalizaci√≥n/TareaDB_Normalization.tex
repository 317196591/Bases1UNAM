\documentclass{article}
\begin{document}
\title{Tarea de Bases de Datos: Normalización}
\author{Daniel Alberto Zarco Manzanares}
\date{\today}
\maketitle

\paragraph{¿Para qué sirve la normalización?}
\paragraph{
Para mejorar el desempeño de una base de datos, así como evitar redundancia en la información que contiene y, en consecuencia, generar condiciones para un mejor diseño, el analista de sistemas debe conocer las formas de normalización y condiciones en las que la desnormalización es recomendable.
}
\paragraph{
Primera forma normal (1FN)
1) Todos los atributos, valores almacenados en las columnas, deben ser indivisibles.
2) No deben existir grupos de valores repetidos.
El valor de una columna debe ser una entidad atómica, indivisible, excluyendo así las dificultades que podría conllevar el tratamiento de un dato formado de varias partes.
}
\paragraph{
Segunda forma normal (2FN)
Una tabla se encuentra en 2FN cuando está en 1FN y no contiene dependencias parciales. Por consiguiente, una tabla 1FN automáticamente está en 2FN si su clave primaria está basada solamente en un atributo simple. Una tabla en 2FN aún puede contener dependencias transitivas.
}
\paragraph{
Tercera forma normal (3FN)
Una tabla se encuentra en 3FN si está en 2FN y no contiene dependencias transitivas, lo cual significa que las columnas que no forman parte de la clave primaria deben depender sólo de la clave, nunca de otra columna no clave.
}
\begin{thebibliography}{a}
\bibitem{pradery} \textsc{Coronel, Carlos and Morris.},
\textit{Database Systems: Design, Implementation and Management}
11ª ed. Pearson,  2016  
\end{thebibliography}
\end{document}
