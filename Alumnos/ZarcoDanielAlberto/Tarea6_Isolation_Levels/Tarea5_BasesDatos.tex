\documentclass{article}
\begin{document}	
\title{Homework 6: Isolation's levels}
\author{Daniel Alberto Zarco Manzanares}
\date{\today}
\maketitle
\paragraph{Isolation levels\\\\}
The isolation level specifies how transactions that modify the database are handled. By default, the default object gateway is used. Not all database types support each isolation level. Some database providers use different names for isolation levels.
Queries that are executed by reports and analysis are intended to be read-only operations. Queries are executed with a unit of work on the data source as a transaction with a default or administrator-defined isolation level. Report authors should not assume that queries that execute stored procedures commit all data written by the procedure. In some environments, changes made by a procedure may be confirmed by database characteristics. A stored procedure marked with write access in Framework Manager commits changes but can only be used with Event Studio.
If you need to run specific queries with different isolation levels, you must define different database connections.\\\\
\begin{tabular}{|c|c|c|c|c|}
Isolation Level & Dirty Read & Non Repeatable Read & Phantom\\
\hline
Read uncommitted & Yes & Yes & Yes\\
Read committed & No & Yes & Yes\\
Repeatable read & No & No & Yes\\
Snapshot & No & No & No\\
Serializable & No & No & No\\
\end{tabular}
\\\\\\Transactions must be run at an isolation level of at least repeatable read to prevent lost updates that can occur when two transactions each retrieve the same row, and then later update the row based on the originally retrieved values. If the two transactions update rows using a single UPDATE statement and don't base the update on the previously retrieved values, lost updates can't occur at the default isolation level of read committed.
\begin{thebibliography}{2}
\bibitem{lamport94}
  Microsoft Docs,
  \textit{Understanding isolation levels},
  https://docs.microsoft.com/en-us/sql/connect/jdbc/understanding-isolation-levels?view=sql-server-ver15,
  Consulted in {\today}
 \end{thebibliography}

\end{document}
