\documentclass{article}
\begin{document}
\title{Homework 7: Third normal form from sport class's dates exercise}
\author{Daniel Alberto Zarco Manzanares}
\date{\today}
\maketitle
\paragraph{Introduction\\\\}
Following class exercise we transform initial table that describe a class schedule to college.
\paragraph{First table\\\\}
\begin{tabular}{|c|c|}
Activity Code & Activity name\\
\hline
1 & Pilates\\
2 & Fitness\\
3 & Yoga\\
4 & Gym\\
\end{tabular}
\paragraph{Second table\\\\}
\begin{tabular}{|c|c|}
Teacher id & Teacher name\\
\hline
12345 & Juan\\
76454 & Javier\\
88903 & Lidia\\
\end{tabular}
\paragraph{Third table\\\\}
\begin{tabular}{|c|c|c|c|c|c|}
Activity number & Teacher id & Place & Date & Start hour & Finish hour\\
\hline
1 & 12345 & Pavilion & 9/9/2007 & 10:00 & 11:00\\
2 & 76454 & Pavilion & 9/9/2007 & 10:00 & 11:00\\
1 & 12345 & Pavilion & 11/9/2007 & 9:30 & 11:00\\
1 & 12345 & Pavilion & 15/9/2007 & 12:00 & 13:00\\
3 & 76454 & Multipurpose room & 15/9/2007 & 9:00 & 10:00\\
4 & 12345 & Multipurpose room & 1/10/2007 & 12:00 & 13:00\\
3 & 76454 & Multipurpose room & 15/9/2007 & 11:00 & 12:00\\
4 & 88903 & Pavilion & 1/10/2007 & 12:00 & 14:00\\
2 & 88903 & Pavilion & 9/9/2007 & 10:00 & 12:00\\
1 & 76454 & Multipurpose room & 9/9/2007 & 10:00 & 12:00\\
\end{tabular}
\paragraph{Conclusion\\\\}
Third normalization form can help us to have organized and structured table from any data to give us. In this sense, memory's analysis give us quickly query at moment of management the table. This analysis was produces by dependency's analysis.  

\end{document}
